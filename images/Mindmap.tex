\documentclass[tikz, border=10pt]{standalone}
\usepackage[T1]{fontenc} % Verwendung von Vektorschriften
\usepackage[ngerman]{babel} % Deutsche Rechtschreibung und Silbentrennung
\usepackage{tikz}
\usetikzlibrary{mindmap, shapes.geometric} % Import Bibliothek

% Farbdefinitionen
\definecolor{MindMapBlue}{RGB}{0, 105, 180}
\definecolor{MindMapGreen}{RGB}{76, 153, 0}
\definecolor{MindMapRed}{RGB}{237, 28, 36}
\definecolor{MindMapYellow}{RGB}{255, 242, 0}
\definecolor{MindMapOrange}{RGB}{255, 127, 39}
\definecolor{MindMapPurple}{RGB}{150, 111, 214}
\definecolor{MindMapGray}{RGB}{195, 195, 195}
\definecolor{MindMapWhite}{RGB}{255, 255, 255}
\definecolor{MindMapBlack}{RGB}{0, 0, 0}

\begin{document}
\begin{tikzpicture}[mindmap, grow cyclic,
                    every node/.style={concept, fill=none, text width=2.5cm, font=\small, align=center, line width=0.5pt},
                    level 1/.append style={sibling angle=360/4},
                    level 2/.append style={level distance=4cm, sibling angle=45},
                    level 3/.style={sibling angle=45}, % Unterunterknoten
                    edge from parent/.style={concept color=MindMapBlack, line width=0.5pt},
                    text only/.style={draw=none, rectangle, text=MindMapBlack}]

% Hauptknoten
\node[text width=4cm, font=\large\bfseries]{Mindmap}
    child [concept color=MindMapGreen] { node {Struktur einer Mindmap}
        child { node [text only] {Zentraler Knoten}}
        child { node [text only] {Verzweigungen}}
        child { node [text only] {Hierarchische Struktur}}
    }
    child [concept color=MindMapRed] { node {Visuelle Gestaltung}
        child { node [text only]  {Visuelle Elemente}}
        child { node [text only]  {Gestaltungsfreiheit}}
    }
    child [concept color=MindMapBlue] { node {Anwendung}
        child { node [text only]  {Brainstorming}}
        child { node [text only]  {Planung}}
        child { node [text only]  {Problemlösung}}
        child { node [text only]  {Notizen}}
        child { node [text only]  {Lernen} }
    }
    child [concept color=MindMapOrange] { node {Vorteile}
        child { node [text only] {Förderung des kreativen Denkens}}
        child { node [text only] {Ideenfindung}}
    };
\end{tikzpicture}
\end{document}
