\documentclass[tikz, border=10pt]{standalone}
\usepackage[T1]{fontenc} % Verwendung von Vektorschriften
\usepackage[ngerman]{babel} % Deutsche Rechtschreibung und Silbentrennung
\usepackage{tikz}
\usetikzlibrary{mindmap, shapes.geometric} % Import Bibliothek

% Farbdefinitionen
\definecolor{MindMapBlue}{RGB}{0, 105, 180}
\definecolor{MindMapGreen}{RGB}{76, 153, 0}
\definecolor{MindMapRed}{RGB}{237, 28, 36}
\definecolor{MindMapYellow}{RGB}{255, 242, 0}
\definecolor{MindMapOrange}{RGB}{255, 127, 39}
\definecolor{MindMapPurple}{RGB}{150, 111, 214}
\definecolor{MindMapGray}{RGB}{195, 195, 195}
\definecolor{MindMapWhite}{RGB}{255, 255, 255}
\definecolor{MindMapBlack}{RGB}{0, 0, 0}

\begin{document}
\begin{tikzpicture}[mindmap, grow cyclic,
                    every node/.style={concept, fill=none, text width=2.5cm, font=\small, align=center, line width=0.5pt},
                    level 1/.append style={sibling angle=360/4},
                    level 2/.append style={level distance=4cm, sibling angle=45},
                    level 3/.style={sibling angle=45}, % Unterunterknoten
                    edge from parent/.style={concept color=MindMapBlack, line width=0.5pt},
                    text only/.style={draw=none, rectangle, text=MindMapBlack}]

% Hauptknoten
\node[text width=4cm, font=\large\bfseries]{Markdown}
    child [concept color=MindMapGreen] {
      node {Leichtgewichtige Auszeichnungssprache}
        child { node [text only] {Ziel: Vereinfachung der Webinhalte-Erstellung} }
    }
    child [concept color=MindMapRed] {
      node {Schlüsselkonzepte}
        child { node {Einfachheit}
          child { node [text only] {Syntax} }
          child { node [text only] {Intuitiv} }
        }
        child { node {Lesbarkeit}
        }
        child { node {Konvertierbarkeit}
        }
        child { node {Flexibilität}
        }
        child { node {Erweiterbarkeit}
        }
    }
    child [concept color=MindMapBlue] {
      node {Anwendung}
        child { node [text only] {Webcontent-Erstellung}
        }
        child { node [text only] {Dokumentation}
        }
        child { node [text only] {Notizen und Bücher}
        }
    }
    child [concept color=MindMapOrange] {
      node {Entwicklung und Geschichte}
        child { node [text only] {John Gruber mit Aaron Swartz} }
        child { node [text only] {2004} }
    };
\end{tikzpicture}
\end{document}
